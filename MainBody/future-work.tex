Although a solution to the problem posted has been developed, there is a large scope for future work based on this report. I have chosen to break this down into 2 key areas.

\section{Navigation Mode}
The proposed multi-point system works in theory, but in practice additional work is required to realise the algorithm. It is not currently possible to emit audio from a single stereo channel, due to limitations in the Java class developed for the prototype. It should be quite trivial to add this functionality, but due to a lack of time, this was not possible.

\section{Detail Mode}
As detail mode was seen as being the most difficult problem solve, it seems natural that it is the problem with the largest scope for future work.

\subsection{Basis Shapes Selection}
In order to continue the work on the basis-shapes proposition (of describing an unknown shape in terms of other, known shapes), research should be done into the optimal choice of basis shapes.

The basis shapes used sections \ref{sec:eftbs} and \ref{sec:hubs} were fairly arbitrary choices --- the ideal basis shapes should be experimentally determined.

Additionally, for the sake of completeness, the basis-shape method should be trialled using Zernike Moments as descriptors.

\subsection{Basis Shape Sonification}
Due to time constraints, a limited amount of sonification was done during the basis shapes experiments. Future work in this department will include choosing the optimal choice of sounds --- be it sine waves of different frequencies, different instruments, or a single instrument with different notes for each shape.
