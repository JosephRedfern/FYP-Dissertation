I believe my reflection on learning can be surmised with two quotes --- the first being a law coined by Douglas Hofstader, that: ``It always takes longer than you expect, even when you take into account Hofstadter's Law'', the second having been said by H. L. Mencken: ``For every complex problem there is an answer that is clear, simple, and wrong.''.

I have enjoyed working on the project immensely, having never undertaken any \textit{serious} research task in the past. With that in mind, if I were to repeat the task, I would have done some things differently.

Firstly, I would have modified the scope of the project to be slightly more realistic, due to a combination of \textbf{not} having previously been aware of Hofstaders Law, and due to me now being more aware of the (current) limits to my knowledge. Before starting the project, I think a part of me felt that I would be able to quickly read some articles on the Matlab website and OpenCV Wiki, and instantly be aware of every nuance of Computer Vision that would be relevant to the project. Were I more aware of my own limitations at the time, given the amount of time available to complete the task, I would have reduced the scope of the project.

Secondly, I would have managed and prioritised my time more effectively. I had agree to take part in several events over the course of the Final Year Project, which took away valuable research and programming time. That is not to say that I regret not locking myself in my room for 12 solid weeks --- but in the future, I intend to schedule and plan my time more effectively, and avoid leaving things to be quite so last minute.
