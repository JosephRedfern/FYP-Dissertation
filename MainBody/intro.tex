\ac{WHO} figures claim that as as of 2012, there are 285 million people suffering from visual imparements~\cite{whoblindness}, 30 million of whom are blind. The \ac{WHO} also state that 90\% of the visually impared live in developing countries. Combined with statistics from the guide dogs for the blind association, who claim that the life-time cost of training and keeping a guide dog is around £50,000, a sad picture is painted the majority who are unable to afford visual aids.

\section{Aims and Goals}
This project aims to research and develop a methods of conveying visual information without the user of the system requiring a functional visual system. The study investigates both navigational and semantic modes of operation, and the different techniques associated with each implementation.

The goal of the project is to provide a functional system, able to be used to effectivley navigate around a room and avoid obstacles without the use of eyes.  

\section{Intended Audience}
The main intended audience for this project is the visually impared. It is anticipated that the ``tech savvy'' visually impared would be interested in trialling the prototypes, both for day-to-day use, and to assist in further development.

The system does is not intended to replace all other forms of visual assistance - rather, it intends to assist those who are unable to afford luxuries such as Guide-dogs or Guide-horses~\cite{guidehorse}. A low-cost system that is capable of assisting a user in navigating a room and detecting obsticles has the potential to be life changing - even if only a small fraction of a normal, functional visual system can be imitated, 0.01\% is infinitley greater than 0.00\%.

\section{Project Scope}
The project has a fairly broad scope, and includes:
\begin{itemize}
    \item \textbf{Image Segmentation} \hfill \\
        This deals with the extracting the input image, and extracting an object of interest
    \item \textbf{Descriptor Extraction} \hfill \\
        This is the extraction of descriptors from the object of interest (obtained from step 1)
    \item \textbf{Descriptor Sonification} \hfill \\
        Sonification of the output from step 2.
\end{itemize}

Components/considerations \textbf{outside} of the scope of this project are:
\begin{itemize}
    \item \textbf{Camera Evauluation} \hfill \\
        Detailed evaluation of the various types of depth-sensing cameras is out of scope of this project. 
    \item \textbf{Robust Testing} \hfill \\
        It is not anticipated that the system described in this report go into immediate production -- in-depth safety testing is out-of-scope. 
    \item \textbf{In-depth UX/UI development} \hfill \\
        The UI should be functional, but UI usability is not the primary focus of the the project. 
\end{itemize}

\section{Approach}
I took a parallel approach while researching my solution to the problem. Rather than carry out work step-by-step, where problems and delays could have had a serious impact on the project schedule, work was carried out asynchronously to help mitigate such risks. 

\section{Assumptions}
Several assumptions have been made during the development of the software.

Firstly, it is assumed that the system will be used indoors. The Asus XTION Pro Live camera~\cite{xtion} camera chosen for use in the project uses \ac{IR} laser light to measure distance. In direct sunlight, the \ac{IR} radiation emmitted by the sun over-powers the \ac{IR} light emitted by the XTION, resulting in in-accurate readings. This is a limitation of a majority of \ac{RGB-D} cameras using structured light or Time-of-flight, although future work could involve the use of a stereoscopic camera setup as a solution to the problem.

Secondly, as the systems described in this report convey information in the form of audio, it is assumed that the user of the system has a functional auditory system. Insert statistics here about deafblind if possible - if not, extrapolate based on percentage of blind and percentage of deaf.

\section{Summary of Outcomes}
The report details several approaches to solving the problem of sonification of image and depth data, for both navigation and semantic understanding purposes. 
